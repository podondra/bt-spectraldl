\chap Machine Learning

Machine learning is subfield of articial inteligence \cite[russell10].
Algorithms in machine learning allow computer programs to automatically
improve through experience.
Widely quoted formal definition is provided in \cite[mitchell97]:
``{\em A computer program is said to learn from
experience E with respect to some class of tasks T and performance measure
P, if its preformance at tasks in T, as measured by P, improves with
experince E.}''.

Usually what is meant by experience are data.
Instead of hardcoding the desired model directly into program, data are feeded
into machine learning algorithm which develops its own model. This is called
{\bf data-driven approch} since it depends on providing the algorithm with
training data \cite[cs231n].

For example, considering the image classification program. It would be
challenging to hardcode into program how general dog looks. Machine learning
algorithm in combination with data-driven approch would take large collention
of dog's images, feed them into program and let the program develop its own
notion of what a dog looks like.

\sec Deep Learning

Machine learning offers varied range of approches.
One of them is deep learning which tries to build these concepts from
hierarchy of many simple concepts and because this graph is deep this
approch is called deep learning \cite[goodfellow16].

\chap Frameworks for Deep Learning

\sec Theano

TODO \cite[theano]
