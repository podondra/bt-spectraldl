\chap Machine Learning

Machine learning is subfield of artificial intelligence \cite[russell10].
Algorithms in machine learning allow computer programs to automatically
improve through experience.
Widely quoted formal definition is provided in \cite[mitchell97]:
``{\em A computer program is said to learn from
experience E with respect to some class of tasks T and performance measure
P, if its performance at tasks in T, as measured by P, improves with
experience E.}''.

Usually what is meant by experience is knowledge acquired by extracting
patterns from data \cite[goodfellow16].
Instead of hardcoding the desired model directly into program, data are feeded
into machine learning algorithm which develops its own model. This is called
{\bf data-driven approach} since it depends on providing the algorithm with
training data \cite[cs231n].

For example, considering the image classification program. It would be
challenging to hardcode into program how general dog looks. Machine learning
algorithm in combination with data-driven approach would take large collection
of dog's images, feed them into program and let the program develop its own
notion of what a dog looks like.

\sec Deep Learning

Machine learning offers varied range of approaches.
One of them is deep learning which tries to build these concepts from
hierarchy of many simple concepts and because this graph is deep this
approach is called deep learning \cite[goodfellow16].

This concept of representing complicated structure in terms of other simpler
representation is main feature of deep learning, which distinguish it from other
branches of machine learning \cite[goodfellow16].
Deep learning algorithm can recognize images by firstly identifying corners
then building contours from them and finally construct the whole object.
It is extremely difficult to go from raw pixel data straight to the object
in image recognition.

\chap Frameworks for Deep Learning

This chapter makes overview of some open source deep learning frameworks.

Because astronomy need to analyze indeed vast avalanche of data the framework
should be scalable across multiple CPUs or GPGPUs. Secondly, its source should
be open source with good community support and documentation. Next, general
concepts behind the framework's API are studied in order to recognize poteniial
problems as soon as possible.

\sec Theano

Theano is Python library for working with mathematical expression.
Even though Theano is not developed as deep learning framework it can be
consider as it is because primarily usage is to implement
mathematical models in a symbolic way and then used them in machine or
deep learning. \cite[theano]

Theano is an open source software and is licensed under BSD license. Its source
code is versioned in repository on
GitHub\urlnote{https://github.com/Theano/Theano/}.

Main features of Theano library are in its language to represent mathematical
expressions, a compiler the creates code to compute these expressions and
library which can evaluate them.
This structure allows to optimize code for both CPUs and GPGPUs
which are using CUDA. \cite[theano]

Mathematical expressions have structure of directed, acyclic graph. There are
two kind of nodes. {\bf Variable} nodes which represent the data and
{\bf apply} nodes which stand for mathematical operators. \cite[theano]

\sec TensorFlow

Probably the most popular machine learning library nowadays is TensorFlow.
This library is developed by Google.
It is licensed under Apache 2.0 open source license and
TensorFlow code is available on
Github\urlnote{https://github.com/tensorflow/tensorflow}. \cite[tensorflow]

Computation in TensorFlow is described by a directed graph that is set of
nodes. Each node represents an operation and has zero or more inputs and zero or
more outputs. Values that flow along a graph edges are called {\bf tensors}.
Tensors are arbitrary multidimensional data.
These graphs can be constructed through frontends.
Currently implemented frontends support C++ and Python. \cite[tensorflow]

\chap Astroinformatics

\sec FITS data format

Flexible Image Transport System (FITS) is standard data format used in
astronomy. It is used to transport not only image data but also
astronomical spectra, data cubes, table and header with keywords providing
descriptive information about the data. \cite[fits-support-office]
