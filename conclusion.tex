\chap Conclusion

The goal of this thesis was to study deep learning in context of
astroinformatics and implement program for astronomical electromagnetic spectra
analysis. Currently the work introduces to contect of deep learning as branch
of machine learning. Main concept of this approch are explained.

Then Theano framework as Python computational framework is analyzed.
Conslusions of its analysis with respect to astronomical data are
promissing.
Since on top of it cuDNN framework may be used to implement deep learning
techniques as well as this implementation can be scaled to general purpose
graphical processing units.

Finaly, FITS file format, which is the main representation of astronomical
spectra, is described.
Moreover, work mentions part Python library called astropy.io.fits
as interesting solution for parsing FITS files.

In the future it is vital to deepen overview of availabel framework as well
as show some preprocessing methods for astronomical spectra preparation.
After that stage is done actual program, which uses deep learning concept
intoroduced in very begging of this work, should be developed and tested on
real data from spectrographs.
