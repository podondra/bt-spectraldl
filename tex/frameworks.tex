\chap Deep Learning Frameworks

There are tens of deep learning frameworks.
This chapter makes overview of such frameworks in order to choose the
most suitable for purposes of this thesis.
Several aspects have been taken into account:

\begitems
* great documentation,
* open source project,
* long term support,
* scalable and GPGPU ready,
* allows fast experiments,
* Python API.
\enditems

The three first points should asure that this work will be maitanable
in future. This work is faced with vast amount of data so it is desirable
to choose scalable and GPGPU ready framework.
On contrary, one goal is to find great deep neural network architecture which
means that fast experimentation is crucial.
Last two points support the goal of this thesis to find the best deep neural
network architecture and because the author comfortable with Python.

NVIDIA's website\urlnote{https://developer.nvidia.com/deep-learning-frameworks}
lists popular deep learning frameworks. These include Caffe, TensorFlow,
Theano, Torch and Keras. Torch is excluded because it does not has Python
bindings \cite[torch].

\sec Caffe

Caffe is BSD-licensed C++ deep learning framework with Python and MATLAB
bindings.
It serves for training and deploying convolutional neural networks and
other deep models.
By separeting model representation from actual implementattion, Caffe allows
seamless switching among CPU, GPGPU and other architectures.
The project is maintained by Berkeley Vision and Lerning Center with help of
community of contibutors on GitHub\urlnote{https://github.com/BVLC/caffe/}.
Documentation and examples are available at Caffe's
website\urlnote{http://caffe.berkeleyvision.org/}. \cite[caffe]

According to experimental results in \cite[benchmark] Caffe has one of the
best performances on both GPGPU and CPU platforms.

\sec TensorFlow

TensorFlow is interface developed by Google for expressing machine learning
algorithms and an implementation for executing such algorithms.
It is licensed under Apache 2.0 open source license and its code is available on
Github\urlnote{https://github.com/tensorflow/tensorflow}.
A computation in TensorFlow is described by a directed graph which can be
executed with little or no change on CPUs or GPUs.
Currently implemented frontends are C++ and Python.
This project has extensive documentation with examples and
tutorial\urlnote{https://www.tensorflow.org/}. \cite[tensorflow]

In order to help develepment and debugging TensorFlow provides visualization
tool called TensorBoard.
This tool can vizualize computation graphs and summary statistics such as
values of loss function, weights and so on.

\sec Theano

Theano is a Python framework for working with mathematical expression.
Even though Theano is not developed as deep learning framework it can be
consider as it is because primarily usage is to implement
mathematical models in a symbolic way and then used them in machine or
deep learning.
Theano is an open source software and is licensed under BSD license.
Its source code is versioned in repository on
GitHub\urlnote{https://github.com/Theano/Theano/}.
Main features are in language to represent mathematical expressions, a compiler to creates code to compute these expressions and library to evaluate them.
This structure allows to optimize code for both CPUs and GPGPUs. \cite[theano]

As stated in \cite[theano] although Theano is developed and mainly used for
reasearch in deep learning it is not a deep learning framework in itself.
Several software packages have been developed to provide higher-level
user interface for instance Keras, Blocks and Lasagne.

\sec Keras

Keras is high-level neural networks API. It is written in Python and it runs
on top of either TensorFlow or Theano.
It focuses on fast prototyping and experimentation.
Because it is interface to TensorFlow or Theano it can use both CPU and GPGPUs.
Its website\urlnote{https://keras.io/} provides good documentation and code is
open source under MIT license on
GitHub\urlnote{https://github.com/fchollet/keras}. \cite[keras]

Recently Keras API will be integrated into TensorFlow 1.2.
Therefore, there will be two separete implementations of the Keras
specification.
One for written in pure TensofFlow and second compatible with
both Theano and TensorFlow. \cite[keras2]

\sec Consideration

TODO
