\input ctustyle % the template is included here
\def\thednum{(\the\chapnum.\the\dnum)}
\input opmac-bib % uncomment this for direct reading of .bib database files 
\input dirtree

\worktype [B/EN] % type: B = bachelor, M = master, D = Ph.D., O = other

\faculty     {F8}
\department  {Department of Theoretical Computer Science}
\title       {Deep Learning in Large Astronomical Spectra Archives}
\author      {Ondřej Podsztavek}
\date        {May 2017}
\supervisor  {RNDr. Petr Škoda, CSc.}
\studyinfo   {Branch of Study: Computer Science} % study programme etc. 
\titleCZ     {Hluboké učení ve velkých archívech astronomických spekter}
\abstractEN  {
    Large astronomical archives, as for example \glref{LAMOST} spectral archive,
    contain plenty of hidden information.
    Deep learning is currently very popular method used
    to gain knowledge from this kind of data.
    This work shows the process of finding emission-line spectra in
    \glref{LAMOST} archive using deep convolutional neural network trained
    on data from Ondřejov 2m telescope.
    Overview of several techniques as spectra preprocessing,
    domain adaptation of Ondřejov data to \glref{LAMOST} resolution,
    dimensionality reduction, architecture
    and training of two deep neural networks are presented.
    Finally, discovered objects with interesting physical nature deserving
    further detailed analysis are discussed.
}
\abstractCZ  {
    Velké astronomické archívy, jako například spektrální archív \glref{LAMOST},
    obsahují řadu skrytých informací.
    Hluboké učení je jednou z nejpopulárnějších dnes používaných metod
    pro získávání znalostí z tohoto druhu dat.
    Tato práce popisuje proces hledání spekter s emisními čarami v archívu
    \glref{LAMOST} za použití hluboké konvoluční neuronové sítě naučené
    na datech z ondřejovského 2m teleskopu.
    Práce popisuje několik metod jako je předzpracování spekter,
    doménová adaptace ondřejovských dat na rozlišení archívu \glref{LAMOST},
    redukce dimenzionality, návrh a učení dvou neuronových sítí.
    V závěru práce je diskuze objevených objektů se zajímavou fyzikální
    podstatou, které vyžadují další detailní analýzu.
}
\keywordsEN  {
    deep learning, neural networks, dimensionality reduction, domain adaptation,
    astroinformatics, astronomy, LAMOST, TensorFlow
}
\keywordsCZ  {
    hluboké učení, neuronové sítě, redukce dimenzionality, doménová adaptace,
    astroinformatika, astronomie, LAMOST, TensorFlow
}

\thanks      {
    This work would have been impossible without the support of Petr Škoda,
    his domain expertise and enthusiasm. I would like to thank to my family
    for their extraordinary support during my bachelor's degree.
    This research was supported by the grant COST LD-15113 of the Ministry of
    Education Youth and Sports of the Czech Republic.
}
\declaration {
    I hereby declare that the presented thesis is my own work and that I have
    cited all sources of information in accordance with the Guideline for
    adhering to ethical principles when elaborating an academic final thesis.

    I acknowledge that my thesis is subject to the rights and obligations
    stipulated by the Act No. 121/2000 Coll., the Copyright Act, as amended.
    In accordance with Article 46(6) of the Act, I hereby grant a nonexclusive
    authorization (license) to utilize this thesis, including any and all
    computer programs incorporated therein or attached thereto and all
    corresponding documentation (hereinafter collectively referred to as the
    “Work”), to any and all persons that wish to utilize the Work. Such persons
    are entitled to use the Work in any way (including for-profit purposes)
    that does not detract from its value. This authorization is not limited in
    terms of time, location and quantity. However, all persons that makes use
    of the above license shall be obliged to grant a license at least in the
    same scope as defined above with respect to each and every work that is
    created (wholly or in part) based on the Work, by modifying the Work, by
    combining the Work with another work, by including the Work in a collection
    of works or by adapting the Work (including translation), and at the same
    time make available the source code of such work at least in a way and scope
    that are comparable to the way and scope in which the source code of the
    Work is made available.

    In Prague on 16th May 2017
    \signature
}

%%%%% <--   % The place for your own macros is here.

%\draft            % uncomment this if the version of document is working only
%\linespacing=1.7  % uncomment this if you need more spaces between lines
                   % warning: this works only when \draft is activated
%\savetoner        % turns off the lightBlue backround of tables and
                   % verbatims, only for \draft version
%\blackwhite       % use this if you need really blackwhite thesis
%\onesideprinting  % use this if you really don't use duplex printing

\input glosdata

\makefront

\input introduction
\input spectral-data
\input machine-learning
\input transfer-learning
\input frameworks
\input ondrejov
\input preprocessing
\input dimensionality
\input lamost
\input conclusion

\bibchap
\usebib/c (simple) references

\input appendix

\bye
