\chap Preprocessing

This chapter describes all the action taken on the data from Ondřejov
dataset before processing then through a neural network.

\sec Knowledge Transfer

Ondřejov and LAMOST spectrographs have different parameters
(as described in sections \ref[ondrejov-archive] and
\ref[lamost-archive]).
Thus, spectra from the two archive different.
Concretly, spectra from Ondřejov archive has much more details than
spectra from LAMOST spectrograph
and spectra in Ondřejov archive are stored in air wavelengths
but LAMOST store spectra in vacuum wavelengths.

This thesis is concerned with classifying LAMOST data using
data from Ondřejov.
It aims to extract as much knowledge as possible from Ondřejov data
and apply it to classification of LAMOST archive.
Tranfer learning and domain adaptation
(described in chapter \ref[transfer-learning])
are study area of machine learning which aim to offer methods
for dealing with these problems.

The following sections introduce methods
that were applied to Ondřejov data.
These methods make the two domains more similar and so
they deal with differences between the archives mentioned above.

\secc Wavelength Conversion

Ondřejov and LAMOST archives store wavelengths in air and vacuum
wavelength respectively.
When spectra of same object from the two archives are plotted on each
other they are a bit shifted (see top plot in figure \ref[air2vacuum]).
Therefore, Ondřejov spectra are converted to vacuum wavelengths
according to formulas provided in \cite[air2vacuum]:

$$ \lambda_{v} = n \lambda_{a} \eqmark $$

$$
n = 1 + 8.34254 \cdot 10^{-5} +
    {2.406147 \cdot 10^{-2} \over (130 - s^2)} +
    {1.5998 \cdot 10^{-4} \over (38.9 - s^2)}
\eqmark
$$

$$ s = {10^4 \over \lambda_{a}} \eqmark $$

where $\lambda_{v}$ is vacuum wavelength and $\lambda_{a}$ is
corresponding air wavelength.

The resulting Ondřejov spectrum after air to vacuum conversion
is plotted in bottom plot in figure \ref[air2vacuum].

\midinsert \clabel[air2vacuum]{Air to vacuum conversion}
\picw=15cm \cinspic img/air2vacuum.png
\caption/f Spectrum of object HIP47636 from Ondřejov archive before
and after covnersion of air to vacuum wavelengths in comparison with
spectrum of the same object from LAMOST archive.
\endinsert

\secc Gaussian Blur

According to \cite[szeliski2010] Gaussian filter smooths away high-frequency
detail of images.
Spectrum can be seen as one dimensional image.
Correctly parametrized Gaussina blur applied to Ondřejov spectrum
would reduce the amount of detail
and thus make it more similar to spectrum from LAMOST archive.

The equation of a one dimensional Gaussain function:

$$ G(x) = {1 \over \sqrt{2 \pi \sigma^2}} e^{-{x^2 \over 2 \sigma^2}} \eqmark $$

where $\sigma$ is standard deviation in pixels.

This work uses convolution implementation from Python package {\tt astropy}
\cite[astropy] in version 1.3.1.
Value 7 was choosen for standard deviation after vizualizating results of
convolutions with different parameters.

\begtt
from astropy.convolution import Gaussian1DKernel, convolve

gauss_kernel = Gaussian1DKernel(stddev=7)
smoothed_fluxes = convolve(fluxes, gauss_kernel)
\endtt

\midinsert \clabel[gaussian-blur-plot]{Gaussian blur}
\picw=15cm \cinspic img/gaussian-blur.png
\caption/f TODO
\endinsert

\sec Regridding

TODO

\sec Dimensionality Reduction

TODO

\secc PCA

TODO

\midinsert \clabel[pca-plot]{PCA}
\picw=15cm \cinspic img/pca.png
\caption/f TODO
\endinsert

\secc t-SNE

TODO

\midinsert \clabel[tsne-plot]{TODO}
\picw=15cm \cinspic img/tsne.png
\caption/f TODO
\endinsert

\sec Dataset Split

\midinsert \clabel[dataset-split-table]{Dataset split table}
\ctable{lrrr}{
    \hfil set  & emission & absorption & double-peak \hfil \crl \tskip4pt
    train      &     3817 &       4393 & 1104 \cr
    validation &      954 &       1099 &  276 \cr
    test       &      530 &        611 &  153 \cr
}
\caption/t TODO
\endinsert

\sec SMOTE Balancing

TODO
