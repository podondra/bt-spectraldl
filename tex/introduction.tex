\chap Introduction

Astronomy and science in general are transformed by exponential growth
of data which provides huge opportunities for discovery.
Modern large telescopes like large mosaics of \glref{CCD} chips are prodicing 
terabytes of raw data per night.
For example the world largest spectrograph of \glref{LAMOST} telescope
is acquiring 4000 spectra per single exposure.
Astromy is therefore facing an avalanche of data that can be process
only with with sophisticated and innovative approches.

Recent advance in deep nets have brought breakthroughs in
processing images, video, speech and audio.
Deep learning discovers intricate structure in large datasets.
Thus deep networks should help to make new discoveries in astronomy.

The goal of this work is to identify emission-line spectra
in the \glref{LAMOST} spectral survey archive using deep neural network
which is trained on spectra from Ondřejov archive.

This work starts with introduction to spectral data
in~chapter~\ref[spectral-data].
Chapters~\ref[machine-learning] and \ref[transfer-learning] introduce
basics of machine, deep and tranfer learning.
In~chapter~\ref[frameworks] is survey of currently available Python frameworks
for deep learning.
Chapter~\ref[ondrejov] describes the creation and properties of Ondřejov
dataset which is used for network training.
Chapter~\ref[preprocessing] describes all preprocessig methods
applied to the data.
Visualizations of Ondřejov and \glref{LAMOST} data space are shown
in~chapter~\ref[dimensionality].
The last chapter~\ref[lamost] presents architecture of the deep neural network,
its training and classification results of spectra from \glref{LAMOST}.
