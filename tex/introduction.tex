\chap Introduction

Astronomy has always been one of the most progressive sciences. Nowadays it is
going through a period of explosive growth with amazing discoveries. The modern
spectrographs and telescopes are producing enormous data every night. All these
data provide great opportunity for discovery, but no one can process them
in full. Therefore, intelligent algorithms are needed to analyze and extract
interesting content from this data avalanche.

Astroinformatics is interdisciplinary science between astronomy and
computer science which aims to provide astronomers with tools to analyze and
understand these complex data. This includes distributed database queries,
machine learning based data mining, etc.

This work analyze usage of deep learning methods in astronomy.
Deep learning is one of branches of machine learning. In general, machine
learning is computer science subfield which gives computer program ability to
learn. Work also involves study of different data preparation methods.
Primary focused is on algorithms suitability for astronomical electromagnetic
spectra and their computation efficiency.
