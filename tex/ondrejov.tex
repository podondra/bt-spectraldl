\label[ondrejov]
\chap Classifying Ondřejov Archive

This chapter discusses collection and classification
of spectral dataset from Ondřejov \glref{CCD}700 archive.
Overview of a tool used for classification is made
and then it shows dataset's properties and statistics.

\sec Spectral View

Spectral View is web browser tool for classification of Ondřejov \glref{CCD}700
spectral archive.
It was implemented to support creation of dataset for this work.
It is written in Python using
Tornado Web Framework\urlnote{http://www.tornadoweb.org/en/stable/}
and Motor\urlnote{https://motor.readthedocs.io/en/stable/}
which is asynchronous driver for MongoDB.
Visualizations are created with D3.js\urlnote{https://d3js.org/}.
Source code is available on
GitHub\urlnote{https://github.com/podondra/spectralview}
and documentation is available on
Read the Docs\urlnote{http://spectralview.readthedocs.io/}.

Spectra are divided into emission, absorption, double-peak and unknown
classes according to shape of spectral line at H$\alpha$
(see figure~\ref[classes]).

\midinsert \clabel[classes]{Classes in Ondřejov dataset}
\picw=15cm \cinspic img/classes.png
\caption/f Examples from Ondřejov dataset of spectra from emission,
absorption and double-peak classes.
\endinsert

Data import is done through
a \glref{SSAP} Service\urlnote{http://www.ivoa.net/documents/SSA/}.
These spectra are then split into corresponding classes by a user
who is presented with visualization of full spectrum,
zoomed H$\alpha$ spectral line and Gaussian convolved H$\alpha$.
Classified spectra can be viewed and move between classes on demand.

\sec Ondřejov Dataset

The resulting dataset consists of 13335 spectra.
Structure of the dataset is shown in~table~\ref[dataset-table].
Emission and absorption spectra are the most significant part of this dataset.
There are 45.77\% spectra with absorption line in H$\alpha$
and 39.75\% with emission line.
But there are also 11.50\% spectra with double-peak in H$\alpha$
plus 2.98\% spectra which cannot be easily classified.

\midinsert \clabel[dataset-table]{Number of spectra in classes}
\ctable{lr}{
    \hfil class & count \hfil \crl \tskip4pt
    emission    & 5\,301 \cr
    absorption  & 6\,103 \cr
    unknown     &    398 \cr
    double-peak & 1\,533 \cr
}
\caption/t Number of spectra in Ondřejov dataset's classes.
\endinsert

All spectra in this dataset contain H$\alpha$
but they have different wavelength starts and ends.
Infimum from all wavelength starts is 6518.4272\,{\AA}
and supremum from all wavelength ends is 6732.7432\,{\AA}.
There is on average one measurement of flux each 0.249\,{\AA}
with small standard deviation 0.0119\,{\AA}.
