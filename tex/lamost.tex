\chap LAMOST Classification

This chapter gathers all knowledge from previous chapters and classify
LAMOST Data Release 1
which contains spectra from pilot and first year surveys
(see table~\ref[lamost]).
Firstly, architecture and training of a deep neural network is described
followed by evaluation and visualizations of results.

\sec Neural Network Architecture

A feedforward neural network and a convolutional neural network were design.
After evaluation on test set the convolutional network
was chosen for final classification.
This design choices follow guide provided by \cite[cs231n].
Next sections describe the architectures.

\secc Feedforward Network

TODO

\secc Convolutional Network

TODO

\secc Dropout

{\bf Dropout} is technique for addressing problem of {\bf overfitting}.
The key idea is to randomly drop units from the neural network
during training.
During training dropout created different smaller networks by blocking
units of a big one.
This prevent units from co-adaption
and at test time the neural network with all units but smaller weights
is average of all the smaller networks from training.
\cite[dropout]

\sec Training the Networks

TODO

\secc Dataset Split

This section shows how the Ondřejov dataset is split into
training, validation and test set.
These sets are required for training, tuning and evaluating any neural network.
All split are done as stratified sampling so that the distribution of
samples' number in a class is kept across sets.

Test set on which neural networks are evaluated contains 10\% of all data.
Validation set which serves for hyperparameter and architecture optimization
is 20\% of remaining data.
Training set is composed from the rest of data
and its purpose is for training networks' weights contains 80\% of all data.
Exact numbers of spectra in each set is in table~\ref[dataset-split-table].

\midinsert \clabel[dataset-split-table]{Dataset split table}
\ctable{lrrr}{
    \hfil set  & emission & absorption & double-peak \hfil \crl \tskip4pt
    train      &   3\,817 &     4\,393 & 1\,104 \cr
    validation &      954 &     1\,099 &    276 \cr
    test       &      530 &        611 &    153 \cr
}
\caption/t Exact numbers of samples in train, validation and test set
divided according to emission, absorption and double-peak classes.
\endinsert

\secc SMOTE Balancing

\glref{SMOTE} is shortcut for
{\bf Synthetic Minority Over-sampling Technique}.
It is over-sampling approach in which the minority class is
over-sampled by creating {\bf synthetic} samples.
A new sample is created along line from a sample to its all or any $k$
nearest neighbors from same class.
Difference between feature vectors of sample under consideration
and nearest neighbor is taken.
It is multiply by random number between 0 and 1.
Finally it is added to the sample.
\cite[smote]

Scikit-learn contribution Python package {\bf imbalanced-learn} \cite[imblearn]
in version 0.2.1:

\begtt
import imblearn.over_sampling

N_CLASSES = 3
smote = imblearn.over_sampling.SMOTE()
for _ in range(N_CLASSES - 1):
    X, y = smote.fit_sample(X, y)
\endtt

\sec Results and Visualizations

TODO

\sec Performance and Scalability

TODO
