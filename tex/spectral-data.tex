\chap Spectral Data

This thesis is concerned with classification of star's spectra.
Aim of this chapter is to describe these data.
The following sections introduce spectroscopy, its related concepts,
Ondřejov and \glref{LAMOST} astronomical archives of spectral data.

\sec Astronomical Spectroscopy

Spectroscopy in astronomy is the study of celestial objects' spectra.
The different wavelengths of electromagnetic radiation are spreaded out into
a spectrum. This information is then used to derive chemical composition,
temperature, distance, relative motion and much more. \cite[vogt2001]

\sec Electromagnetic Radiation

Electromagnetic radiation is pair of electric and magnetic
fields that propagate together at the speed of light
($c$~=~299\,792\,458\,ms$^{-1}$).
Visible light, radio waves, X-rays and gamma rays are
examples of electromagnetic radiation. Electromagnetic
spectrum is collective term for known range of
electromagnetic radiation. The electric and magnetic field
oscillates and produces electromagnetic waves. \cite[cosmos]

The electromagnetic wave is describe in terms of frequency
or wavelength:

\begitems    
* Frequency ($f$) is the number of waves per second and
its unit is Hertz.
* Wavelength ($\lambda$) is the distance between successive
crests or troughs in the wave and it is meausred is meters
or astronomy usually uses {\AA}ngström (1\,{\AA}~=~10$^{-10}$\,m).

\enditems

Frequency and wavelength are related by wave equation:

$$ c = \lambda f \eqmark $$

\sec Black Body Radiation

A black body is a hypothetical object which is a perfect
absorber and emitter of radiation over all wavelengths.
The spectral flux distribution of black body's thermal
energy depends on its temperature.
Stars are often modelled as blackbodies in astronomy.
Their spectrum approximates the black body spectrum. \cite[cosmos]


\midinsert \clabel[black-body]{Black body radiation}
\picw=15cm \cinspic img/black-body.png
\caption/f Black body radiation curves of 3000, 4000 and 5000 Kelvins hot
stars.
\endinsert

\sec Spectrum Continuum

Each spectrum is characterized by the continuum and the spectral lines.
The continuum is generally smoothly varying spectrum of light emitted by
a star while spectral lines are peaks reaching out from a continuum.
\cite[continuum]

In order to avoid variations between different observations a contiuum is
usually normalized to serve as refence point to spectral lines.
A smooth curve is fit to the continuum and then the spectrum is divided by
it.
This sets the continuum everywhere to one and allows to measure spectral
lines in consistent way.

\midinsert \clabel[continuum-normalization]{Continuum normalization}
\picw=15cm \cinspic img/continuum-normalization.png
\caption/f Comparison between raw spectrum and corresponding spectrum with
normalized continuum.
\endinsert

\sec Spectral Lines

Spectral lines can be used to identify the chemical
composition of stars. If a light from a star is separeted
with a prism its spectrum of colours is crossed with
discrete lines. This can be also visualized as flux
of particural wavelengths. {\bf Flux} is the total amount of
energy that crosses a unit area per unit time.

There are two types of spectral lines:

\begitems
* emission and
* absorption lines.
\enditems

{\bf Emission line} occurs when atom in a
higher energy level (excited state) return to lower energy
level and releases energy. According to quantum theory
every atom has a unique set of energy levels. Therefore,
the atom can emit electomagnetic radiation of particular
wavelengths equal to the difference between the energy
levels. Energy and wavelength, frequency are related
throught Planck-Einstein relation:

$$ E = h f, \quad E = {h c \over \lambda} \eqmark $$

where $h$~=~6.62607004~$\cdot$~10$^{-34}$\,J\,$\cdot$\,s is
Planck constant. On a graph of wavelengths fluxes emission
lines appear as peaks above the continuum level.

{\bf Absorption lines} are opposite of emissions. They will
appear when there is an absorbing material between the
source and the observer. The material could be outer layers
of a stars or interstellar gas. Atoms will absorb specific
energies from the electromagnetic spectrum specifically to
atom's energy levels. On graph these absorption
features are show below the level of a star's black body
continuum spectrum.

\secc Balmer Lines

The Balmer lines or Balmer series is the name of spectral lines of hydrogen
atom that result from electron transitions between second energy level
and higher levels.
There are four transitions that are in visible wavelength.
These are named H$\alpha$, H$\beta$, H$\gamma$ and H$\delta$.
Because hydrogen is most abundant element Balmer lines are a commonly
observed feature in spectroscopy. \cite[cosmos]

H$\alpha$ is deep-red visible spectral line.
Its wavelength in air is 6562.8\,{\AA}.
This spectral line is created whe electron moves between the second and third
energy level of hydrogen atom.

\midinsert \clabel[spectral-lines]{Spectral lines}
\picw=15cm \cinspic img/spectral-lines.png
\caption/f Specta of gamma Cas and alpha Lyr with emission and absorption
line in H$\alpha$ respectively.
\endinsert

\sec Be Stars

Be stars are a non-supergiant B stars\fnote{Stars from spectral type B are
hydrogen burning stars which have 2 to 16 times the mass fo the Sun
and surface temperatures between 10000 and 30000 K. \cite[habets1981]}
whose spectrum has or had at some time, one or more Balmer lines in emission.
Althougth this definition is not precise it is sufficient for purpose of this
work. \cite[porter2003]

It is believed that Be star consists of fast rotating star
and star disk which may cause the emissions in Balmer lines.
Absorptions and emissions in electromagnetic spectrum of Be star can
periodically vary.

\label[ondrejov-archive]
\sec Ondřejov Archive

TODO spectra from two different detectors.

Archive of Ondřejov observatory spectral data is available at glref{ASU CAS}
Data Center\urlnote{http://voarchive.asu.cas.cz/}.
It currently contains about 17000 specta \cite[voarchive].
Because this archive is {\bf specialized on Be stars observations}
it contains approximately 13000 spectra with H$\alpha$ spectral line.
The data were obtained with Ondřejov Perek 2\,m telescope and coud\`e camera
with focus 700\,mm.
The spectrograph's spectral resolving power is about 13000 in H$\alpha$
\cite[stelweb].

{\bf Spectral resolving power} of a spectrograph is defined as:

$$ R = {\lambda \over \Delta \lambda} \eqmark $$

where $\Delta \lambda$ is {\bf spectrum resolution} which stands for the smallest
difference in wavelengths that can be distinguished at a wavelength of
$\lambda$.

\midinsert \clabel[ondrejov-vs-lamost]{Spectra comparison}
\picw=15cm \cinspic img/ondrejov-vs-lamost.png
\caption/f Spectra of BT CMi star from Onřejov and \glref{LAMOST} archive.
\endinsert

\label[lamost-archive]
\sec LAMOST Archive

The Large Sky Area Multi-Object Fiber Spectroscopic Telescope (\glref{LAMOST})
is telescope located in China with spectral resolving power 500, 1000 or 1500.
\glref{LAMOST} archive contains more than 7 milions of spectra.
The latest observations are available in data release 5
Q1\urlnote{http://dr5.lamost.org/}. \cite[lamost]

\midinsert \clabel[lamost]{LAMOST data releases}
\ctable{lrr}{
    \hfil survey & total spectra \hfil & stars \hfil \crl \tskip4pt
    pilot        & 958944              & 812911      \cr
    first year   & 1701669             & 1529958     \cr
    second year  & 1648485             & 1501002     \cr
    third year   & 1659028             & 1511032     \cr
    forth year   & 1713059             & 1543415     \crl \tskip4pt
    \hfil total  & 7681185             & 6898318     \cr
}
\caption/t Statistics of \glref{LAMOST} data releases according to \cite[lamostdr4].
\endinsert

\sec FITS File Format

Flexible Image Transport System is data format used within
astronomy for transporting, analyzing, archiving scientific
data files. It is design to store data sets consisting of
multidimensiional arrays and two dimensional tables. \cite[fits-support-office]

A FITS file is comprised of segmets called Header/Data
Units (\glref{HDU}s).
The first \glref{HDU} is called the ``Primary \glref{HDU}''.
The primary data array can contain a 1-999 dimensional
array of numbers. A typical primary array could contain
a 1 dimensional {\bf spectrum}, a 2 dimensional image,
a 3 dimensional data cube.

Any number of additional \glref{HDU}s may follow the primary array.
These \glref{HDU}s are referred as ``extensions''. There are three
types of standart extensions currently defined:

\begitems
* Image Extension
* \glref{ASCII} Table Extension
* Binary Table Extension
\enditems

Every \glref{HDU} consists of an \glref{ASCII} formatted ``Header Unit'' and
``Data Unit''.

Each header unit contains a sequence of fixed-length 80
character long keyword record which have form:

\begtt
KEYNAME = value / comment string
\endtt

Non-printing \glref{ASCII} character such as tabs,
carriage-returns, line-feeds are not allowed anywhere in
the header unit.

Note that the data unit is not required. The image pixels
in primary array or an image extension may have one of
5 supported data types:

\begitems
* 8-bit (unsigned) integer bytes
* 16-bit (signed) integer bytes
* 32-bit (signed) integer bytes
* 32-bit single precision floating point real numbers
* 64-bit double precision floating point real numbers
\enditems

The othe 2 standard extensions, \glref{ASCII} tables and binary
tables, contain tabular information organized into rows
and columns. Binary tables are more compact and are faster
to read and write then \glref{ASCII} tables.

All the entries within a column of a tables have the same
datatype. The allowed data formats for an \glref{ASCII} table
column are integer, signe and double precision floating
point value, character string. Binary table also support
logical, bit and complex data formats.
