\chap Spectral Data

This thesis is concerned with classification of star's spectra.
Aim of this chapter is to describe these data.
The following sections introduce spectrography, its related concepts,
Ondřejov and LAMOST astronomical archives of spectral data.

\sec Astronomical Spectroscopy

Spectroscopy studies properties of matter through its
interaction with different frequency components of the
electromagnetic spectrum. Astronomical spectroscopy
measures the spectrum of electromagnetic radiation of
hot celestial objects. This information is then used to
derive chemical composition, temperature,
distance, relative motion and so on.

\sec Electromagnetic Radiation

Electromagnetic radiation is pair of electric and magnetic
fields that propagate together at the speed of light
($c$~=~299792458\,ms$^{-1}$).
Visible light, radio waves, X-rays and gamma rays are
examples of electromagnetic radiation. Electromagnetic
spectrum is collective term for known range of
electromagnetic radiation. The electric and magnetic field
oscillates and produces electromagnetic waves. \cite[cosmos]

The electromagnetic wave is describe in terms of frequency
or wavelength:

\begitems    
* Frequency ($f$) is the number of waves per second and
its unit is Hertz.
* Wavelength ($\lambda$) is the distance between successive
crests or troughs in the wave and it is meausred is meters
or astronomy usually uses Angstrom (1\,\AA~=~10$^{-10}$\,m).
\enditems

Frequency and wavelength are related by wave equation:

$$ c = \lambda f \eqmark $$

\sec Blackbody Radiation

A blackbody is a hypothetical object which is a perfect
absorber and emitter of radiation over all wavelengths.
The spectral flux distribution of blackbody's thermal
energy depends on its temperature.
Stars are often modelled as blackbodies in astronomy.
Their spectrum approximates the blackbody spectrum. \cite[cosmos]


\medskip \clabel[blackbody]{Blackbody radiation}
\picw=15cm \cinspic img/blackbody.png
\caption/f Blackbody radiation curves of 3000, 4000 and 5000 Kelvins hot
stars.
\medskip

\sec Spectral Lines

Spectral lines can be used to identify the chemical
composition of stars. If a light from a star is separeted
with a prism its spectrum of colours is crossed with
discrete lines. This can be also visualized as flux
of particural wavelengths. {\bf Flux} is the total amount of
energy that crosses a unit area per unit time.

There are two types of spectral lines:

\begitems
* emission and
* absorption lines.
\enditems

{\bf Emission line} occurs when atom in a
higher energy level (excited state) return to lower energy
level and releases energy. According to quantum theory
every atom has a unique set of energy levels. Therefore,
the atom can emit electomagnetic radiation of particular
wavelengths equal to the difference between the energy
levels. Energy and wavelength, frequency are related
throught Planck-Einstein relation:

$$ E = h f, \quad E = {h c \over \lambda} \eqmark $$

where $h$~=~6.62607004~$\cdot$~10$^{-34}$\,J\,$\cdot$\,s is
Planck constant. On a graph of wavelengths fluxes emission
lines appear as peaks above the continuum level.

{\bf Absorption lines} are opposite of emissions. They will
appear when there is an absorbing material between the
source and the observer. The material could be outer layers
of a stars or interstellar gas. Atoms will absorb specific
energies from the electromagnetic spectrum specifically to
atom's energy levels. On graph these absorption
features are show below the level of a star's blackbody
continuum spectrum.

\secc H-alpha

H-alpha is deep-red visible spectral line. Its wavelength is 6562.8\,\AA.
This spectral line is created whe electron moves between the second and third
energy level of hydrogen atom.

\medskip \clabel[spectral-lines]{Spectral lines}
\picw=15cm \cinspic img/spectral-lines.png
\caption/f Part of electromagnetic spectrum of blue supergiant star HD
38771 with a lot of absorpsion and some emission lines.
\medskip

\sec Ondřejov Archive

Archive of Ondřejov observatory spectral data is available at ASU CAS Data
Center\urlnote{http://voarchive.asu.cas.cz/}.
It currently contains about 17000 specta \cite[voarchive].
The data were obtained with Ondřejov Perek 2\,m telescope and CCD700 chip.
The spectrograph's resolution power is about 13000 in H-alpha \cite[stelweb].

{\bf Resolution power} of a spectrograph is defined as:

$$ R = {\lambda \over \Delta \lambda} \eqmark $$

where $\Delta \lambda$ is the smallest difference in wavelengths that can be
distinguished at a wavelength of $\lambda$.

\sec LAMOST Archive

The Large Sky Area Multi-Object Fiber Spectroscopic Telescope (LAMOST) is
telescope located in China with spectral resolution power 500, 1000 or 1500.
LAMOST archive contains more than 7 milions of spectra.
The latest observations are available in data release 5
Q1\urlnote{http://dr5.lamost.org/}. \cite[lamost]

\midinsert \clabel[lamost]{LAMOST data releases}
\ctable{lrr}{
    \hfil survey & total spectra \hfil & stars \hfil \crl \tskip4pt
    pilot        & 958944              & 812911      \cr
    first year   & 1701669             & 1529958     \cr
    second year  & 1648485             & 1501002     \cr
    third year   & 1659028             & 1511032     \cr
    forth year   & 1713059             & 1543415     \crl \tskip4pt
    \hfil total  & 7681185             & 6898318     \cr
}
\caption/t Statistics of LAMOST data releases according to \cite[lamostdr4].
\endinsert

\sec FITS File Format

Flexible Image Transport System is data format used within
astronomy for transporting, analyzing, archiving scientific
data files. It is design to store data sets consisting of
multidimensiional arrays and two dimensional tables. \cite[fits-support-office]

\secc HDUs

A FITS file is comprised of segmets called Header/Data
Units (HDUs). The first HDU is called the ``Primary HDU''.
The primary data array can contain a 1-999 dimensional
array of numbers. A typical primary array could contain
a 1 dimensional {\bf spectrum}, a 2 dimensional image,
a 3 dimensional data cube.

Any number of additional HDUs may follow the primary array.
These HDUs are referred as ``extensions''. There are three
types of standart extensions currently defined:

\begitems
* Image Extension
* ASCII Table Extension
* Binary Table Extension
\enditems

\secc Header Units

Every HDU consists of an ASCII formatted ``Header Unit'' and
``Data Unit''.

Each header unit contains a sequence of fixed-length 80
character long keyword record which have form:

\begtt
KEYNAME = value / comment string
\endtt

Non-printing ASCII character such as tabs,
carriage-returns, line-feeds are not allowed anywhere in
the header unit.

\secc Data Units

Note that the data unit is not required. The image pixels
in primary array or an image extension may have one of
5 supported data types:

\begitems
* 8-bit (unsigned) integer bytes
* 16-bit (signed) integer bytes
* 32-bit (signed) integer bytes
* 32-bit single precision floating point real numbers
* 64-bit double precision floating point real numbers
\enditems

The othe 2 standard extensions, ASCII tables and binary
tables, contain tabular information organized into rows
and columns. Binary tables are more compact and are faster
to read and write then ASCII tables.

All the entries within a column of a tables have the same
datatype. The allowed data formats for an ASCII table
column are integer, signe and double precision floating
point value, character string. Binary table also support
logical, bit and complex data formats.
