\label[dimensionality]
\chap Dimensionality Reduction

After the preprocessing methods mentioned in chapter~\ref[preprocessing]
each spectrum is a point in {\bf 140-dimensional space}.
To better understand the data \glref{PCA} and \glref{t-SNE} dimensionality
reduction algorithms are applied to reduce each point to 2-dimensional space
which can be visualized as scatter plot.
Visualization of both Ondřejov and \glref{LAMOST} data are presented.

\sec PCA

{\bf Principal Component Analysis}
is statistical technique to transform a number
of possibly correlated variables into a smaller number of variables
called principal components. \cite[pca]

\midinsert \clabel[pca-ondrejov]{PCA of Ondřejov dataset}
\picw=15cm \cinspic img/pca-ondrejov.png
\caption/f On the left size is scatter plot with first two principal
components of the original Ondřejov dataset.
The plot more that difference in shape show difference in fluxes values.
Right plot displays result of \glref{PCA} applied to scaled variant
of the dataset.
After such preprocessing method absorption spectra are oriented in left part
of the plot, emission spectra are on the other side and double-peak spectra
are spread across the whole plot.
\endinsert

Left plot in figure~\ref[pca-ondrejov] displays the first
and the second principal
component of Ondřejov dataset which is not standardized.
After further analysis the plot express more the intensities in spectra
fluxes than the shape.
The points far from the dense area in middle-left have high fluxes.
But in this work the interesting feature is shape
not the difference between in fluxes.

The plot on right side in figure~\ref[pca-ondrejov] shows Ondřejov dataset
with suppressed fluxes.
The scatter plot then expresses the structure of the dataset clearly.
Absorption spectra are on the left side
while emission spectra are mostly on the other side.
Double-peak spectra are spread across the whole plot.
This may implies that the emission and absorption spectra can be separable
by linear classifier.
But the double-peak spectra are mixed up with the other classes so
hopefully a deep neural network can find a representation in which all
classes are separable.

Figure \ref[pca-lamost] shows \glref{PCA} applied to sample of \glref{LAMOST}
spectra plotted with all spectra from Ondřejov.
The plot uncovers that most of spectra in \glref{LAMOST} have absorption lines.
A lot of points are not similar to any Ondřejov data.
But there are probably also emission and double-peak spectra.
This is expected behavior because Ondřejov archive is focused on observing
emission and double-peak spectra (see section~\ref[ondrejov]) while
\glref{LAMOST} observation is not specialized.

\midinsert \clabel[pca-lamost]{PCA of LAMOST sample}
\picw=15cm \cinspic img/pca-lamost.png
\caption/f The first two principal components of 50\,000 random spectra
from \glref{LAMOST} data release 1 and whole Ondřejov dataset.
Majority of \glref{LAMOST} spectra seems to be absorptions but there are some
point near to double-peak and emission spectra from Ondřejov.
\endinsert

\sec t-SNE

{\bf T-Distributed Stochastic Neighbor Embedding}
visualizes high-dimensional data
by giving each point a location in a two or three dimensional map.
It is kind of Stochastic Neighbor Embedding.
Unlike \glref{PCA}
(which is linear technique that focus on keeping the low-dimensional
representations of dissimilar points far apart)
\glref{t-SNE} is capable of capturing both global and local structure.
\cite[tsne]

Visualization of Ondřejov dataset with \glref{t-SNE}
is in figure~\ref[tsne-ondrejov].
Same as experiments in \cite[tsne],
dataset is firstly reduced to 30 dimensions with \glref{PCA}
and then reduce to two dimensions using
Scikit-learn implementation of \glref{t-SNE}.
Perplexity, which is \glref{t-SNE} parameter, is set to value 40.

Figure~\ref[tsne-ondrejov] of \glref{t-SNE} visualization is very similar to
\glref{PCA} scatter plot~\ref[pca-ondrejov].
Absorption spectra are on the left side and emission spectra are
on right side.
Double-peak spectra are more oriented in the middle
but they do not form their own cluster.

\midinsert \clabel[tsne-ondrejov]{t-SNE Ondřejov visualization}
\picw=15cm \cinspic img/tsne-ondrejov.png
\caption/f Visualization of Ondřejov dataset with \glref{t-SNE}.
The input data are standardized (see section~\ref[scaling])
and reduced to 30 dimensions
with \glref{PCA}.
\endinsert

Result of \glref{t-SNE} application on \glref{LAMOST} data
with same parameterization is in
visualization~\ref[tsne-lamost].
There are 5\,000 random data points from \glref{LAMOST}
and 1\,000 random points from Ondřejov.
Data are firstly standardized and reduced to 30 dimensions with \glref{PCA}.

The scatter plot shows a big cluster with absorption spectra on left
and smaller cluster of emission spectra on right.
Unlike in figure~\ref[pca-lamost] there seems to be no spectra
from \glref{LAMOST} too far away from Ondřejov spectra.

\midinsert \clabel[tsne-lamost]{t-SNE of LAMOST sample}
\picw=15cm \cinspic img/tsne-lamost.png
\caption/f Visualization of 5\,000 spectra from \glref{LAMOST} data release 1
and 1\,000 spectra from Ondřejov reduce to two dimensions with \glref{t-SNE}.
\endinsert
