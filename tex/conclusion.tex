\chap Conclusion

The goal of this thesis was to study deep learning in context of
astroinformatics and implement program for astronomical electromagnetic spectra
analysis.

Currently the work introduces the context of machine learning and deep learning
as its branch. Specific approaches of deep learning are distinguished.

Then, Theano and TensorFlow frameworks were analyzed. Different features of these
computational libraries as licensing, community and concepts are studied in
order to make conclusions with respect to astronomical data.

Finally, FITS file format, which is the main representation form of astronomical
spectra, is described.

In the future it is vital to deepen overview of available framework as well
as show some preprocessing methods for astronomical spectra preparation.
Moreover, astronomical concept of Virual Observatory and its protocols should
be mentioned.

After that stage is done actual program, which uses deep learning concepts,
should be developed and tested on both well-known testing dataset form UTI
Machine Learning Repository\urlnote{https://archive.ics.uci.edu/ml/index.html}
and real data from spectrographs.
Benchmark testing should be carried out on both CPUs and GPGPUs.

Final step that has to be made it to integrate implemented algorithm into VO
Cloud infrastructure.
