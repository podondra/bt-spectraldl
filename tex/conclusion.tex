\chap Conclusion

Goal of this work was to classify emission-line spectra in \glref{LAMOST}
using deep neural network trained on spectra from Ondřejov.

Three classes of absorption, emission and double-peak spectra were chosen.
Training Ondřejov dataset was created containing 13335 spectra.
Then it was found that to transfer Ondřejov spectra to
\glref{LAMOST} domain wavelength conversion
and Gaussian blur with standard deviation of value 7
should be applied to all spectra from Ondřejov.
Preprocessing techniques as regridding and spectra scaling are used
to make the data suitable for neural networks learning.

\glref{PCA} and \glref{t-SNE} dimensionality reduction algorithms have shown
that absorption and emission spectra should be easily separable
but double-peak spectra require more sophisticated method as deep learning.

Deep convolutional network was evaluated as best choice of architecture for
spectra classification. It architecture and training is described.
Finally results of classification with the network are presented.
The \glref{LAMOST} data release 1 containing 2\,202\,000 spectra was reduced to
303\,905 candidates (emissions and double-peaks).
Although these candidates contains a lot of interesting emission-line
spectra deserving further analysis
there are plenty of noisy misclassified spectra especially in double-peak
predictions.

In future it would be possible to reduce the number of noisy misclassified
spectra either by extension of dataset
or by other method which could identify noisy spectra.
