\chap Deep Learning Frameworks

This chapter makes overview of some open source deep learning frameworks.

Because astronomy need to analyze indeed vast avalanche of data the framework
should be scalable across multiple CPUs or GPGPUs. Secondly, its source should
be open source with good community support and documentation. Next, general
concepts behind the framework's API are studied in order to recognize poteniial
problems as soon as possible.

\sec Theano

Theano is Python library for working with mathematical expression.
Even though Theano is not developed as deep learning framework it can be
consider as it is because primarily usage is to implement
mathematical models in a symbolic way and then used them in machine or
deep learning. \cite[theano]

Theano is an open source software and is licensed under BSD license. Its source
code is versioned in repository on
GitHub\urlnote{https://github.com/Theano/Theano/}.

Main features of Theano library are in its language to represent mathematical
expressions, a compiler the creates code to compute these expressions and
library which can evaluate them.
This structure allows to optimize code for both CPUs and GPGPUs
which are using CUDA. \cite[theano]

Mathematical expressions have structure of directed, acyclic graph. There are
two kind of nodes. {\bf Variable} nodes which represent the data and
{\bf apply} nodes which stand for mathematical operators. \cite[theano]

\sec TensorFlow

Probably the most popular machine learning library nowadays is TensorFlow.
This library is developed by Google.
It is licensed under Apache 2.0 open source license and
TensorFlow code is available on
Github\urlnote{https://github.com/tensorflow/tensorflow}. \cite[tensorflow]

Computation in TensorFlow is described by a directed graph that is set of
nodes. Each node represents an operation and has zero or more inputs and zero or
more outputs. Values that flow along a graph edges are called {\bf tensors}.
Tensors are arbitrary multidimensional data.
These graphs can be constructed through frontends.
Currently implemented frontends support C++ and Python. \cite[tensorflow]

\sec Caffe

TODO.

\sec Keras

TODO.

\chap Spectral Data

TODO. Preview and analysis of spectra as data.

\sec Ondřejov Archive

TODO.

\sec LAMOST Archive

TODO.

\sec FITS Data Format

Flexible Image Transport System (FITS) is standard data format used in
astronomy. It is used to transport not only image data but also
astronomical spectra, data cubes, table and header with keywords providing
descriptive information about the data. \cite[fits-support-office]

\chap Classifying Ondřejov Archive

TODO.

\sec Spectral View

TODO.

\chap Dimensionality Reduction

TODO.

\sec PCA

TODO.

\sec t-SNE

TODO.

\chap Domain Adaptation

TODO. \url{http://docs.astropy.org/en/stable/convolution/kernels.html}.

\sec Gaussian Convolution

TODO.

\sec Box Blur

TODO.

\chap LAMOST Classification

TODO.

\sec Neural Network Architecture

TODO.

\sec Training the Network

TODO.

\sec Results and Visualizations

TODO.

\sec Performance and Scalability

TODO.
