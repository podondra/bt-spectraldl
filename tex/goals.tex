\chap Goals

First goal of thesis' theoretical part is to introduce deep learning concepts
in context of astronomical data. This includes definitions of terms frequently
used in deep learning.  
Second goal is to analyze some preprocessing algorithm and their application
on astronomical spectra. Main focus should be on dimension reduction.  
Last goal of theoretical part is to analyze available deep learning frameworks.
These framework includes Theano, ThensorFlow and Caffe.

Practical part of this thesis primarily aims to take the most suitable deep
learning framework, implement program for astronomical spectra
classification and integrate it into VO Cloud infrastructure.  
The program is ought to be highly scalable on both CPUs and GPUs. UCI Machine
Learning Repository dataset should be used to evaluate its performance.  
Finally, data from LAMOST spectrograph database are going to be taken
and data analysis will be
carried out on them in order to find special types of stars.
